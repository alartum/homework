\documentclass[12pt]{article}

\pagestyle{empty}
\usepackage[utf8]{inputenc}
\usepackage[T2A]{fontenc}
\usepackage[russian]{babel}
\usepackage{cmap}
\usepackage{amsthm}
\usepackage{amsmath}
\usepackage{amssymb}
\usepackage{breqn}
%\usepackage{units}
%\usepackage{fancyhdr}
%\usepackage{forloop}
%\usepackage[pdftex]{graphicx}
\relpenalty=10000
\binoppenalty=10000

\renewcommand{\baselinestretch}{1.0}
\renewcommand\normalsize{\sloppypar}

\setlength{\topmargin}{-0.5in}
\setlength{\textheight}{9.1in}
\setlength{\oddsidemargin}{-0.4in}
\setlength{\evensidemargin}{-0.4in}
\setlength{\textwidth}{7in}
\setlength{\parindent}{0ex}
\setlength{\parskip}{1ex}

 \begin{document}

Вычислим значение z по следующей формуле:
\begin{dmath*}
 \frac {\sin x \cdot \cos y} {2x^{2}y^{3} + \ln\left(x + y\right)} \end{dmath*}
Запишем данные в таблицу:

\begin{tabular}{|c|c|}
 \hline
 Величина&Значение\\
 \hline
$x$&$1$\\
\hline
$d_x$&$0.2$\\
\hline
$y$&$2$\\
\hline
$d_y$&$0.3$\\
\hline
\end{tabular}

Найдём погрешность. Для этого построим частные производные.

Найдём частные производные по каждой из переменных.

$\blacktriangleright$Найдём частную производную следующего выражения по x:
                             \begin{dmath*}
                             \left( \frac {\sin x \cdot \cos y} {2x^{2}y^{3} + \ln\left(x + y\right)} \right)_{x}^{\prime}
                             \end{dmath*}

$\vartriangleright$Упростив, получим: 
                             \begin{dmath*}
                             \left( \frac {\sin x \cdot \cos y} {2x^{2}y^{3} + \ln\left(x + y\right)} \right)_{x}^{\prime}
                             \end{dmath*}

$\vartriangleright$Продифференцируем.

Рассматриваем переменную как константу:
\begin{dmath*}
\left(y\right)_{x}^{\prime} = 0
\end{dmath*}Производная переменной дифференцирования:
\begin{dmath*}
\left(x\right)_{x}^{\prime} = 1
\end{dmath*}По формуле производной суммы:
\begin{dmath*}
\left(x + y\right)_{x}^{\prime} = 1 + 0
\end{dmath*}По теореме о производной сложной функции:
\begin{dmath*}
\left(\ln\left(x + y\right)\right)_{x}^{\prime} =  \frac {1} {x + y}  \cdot \left(1 + 0\right)
\end{dmath*}Вычисляя производную степенной функции:
\begin{dmath*}
\left(y^{3}\right)_{x}^{\prime} = 3y^{2}
\end{dmath*}Вычисляя производную степенной функции:
\begin{dmath*}
\left(x^{2}\right)_{x}^{\prime} = 2x
\end{dmath*}Производная константы равна 0:
\begin{dmath*}
\left(2\right)_{x}^{\prime} = 0
\end{dmath*}Из выражения для производной произведения:
\begin{dmath*}
\left(2x^{2}\right)_{x}^{\prime} = 0x^{2} + 2 \cdot 2x
\end{dmath*}Из выражения для производной произведения:
\begin{dmath*}
\left(2x^{2}y^{3}\right)_{x}^{\prime} = 0x^{2} + 2 \cdot 2xy^{3} + 2x^{2} \cdot 3y^{2}
\end{dmath*}По формуле производной суммы:
\begin{dmath*}
\left(2x^{2}y^{3} + \ln\left(x + y\right)\right)_{x}^{\prime} = 0x^{2} + 2 \cdot 2xy^{3} + 2x^{2} \cdot 3y^{2} +  \frac {1} {x + y}  \cdot \left(1 + 0\right)
\end{dmath*}Рассматриваем переменную как константу:
\begin{dmath*}
\left(y\right)_{x}^{\prime} = 0
\end{dmath*}По теореме о производной сложной функции:
\begin{dmath*}
\left(\cos y\right)_{x}^{\prime} = -\sin y \cdot 0
\end{dmath*}Производная переменной дифференцирования:
\begin{dmath*}
\left(x\right)_{x}^{\prime} = 1
\end{dmath*}По теореме о производной сложной функции:
\begin{dmath*}
\left(\sin x\right)_{x}^{\prime} = \cos x \cdot 1
\end{dmath*}Из выражения для производной произведения:
\begin{dmath*}
\left(\sin x \cdot \cos y\right)_{x}^{\prime} = \cos x \cdot 1\cos y + \sin x \cdot -\sin y \cdot 0
\end{dmath*}Находим производную частного:
\begin{dmath*}
\left( \frac {\sin x \cdot \cos y} {2x^{2}y^{3} + \ln\left(x + y\right)} \right)_{x}^{\prime} =  \frac {\left(\cos x \cdot 1\cos y + \sin x \cdot -\sin y \cdot 0\right) \cdot \left(2x^{2}y^{3} + \ln\left(x + y\right)\right) - \sin x \cdot \cos y \cdot \left(0x^{2} + 2 \cdot 2xy^{3} + 2x^{2} \cdot 3y^{2} +  \frac {1} {x + y}  \cdot \left(1 + 0\right)\right)} {\left(2x^{2}y^{3} + \ln\left(x + y\right)\right)^{2}} 
\end{dmath*}$\vartriangleright$Упростим полученную производную.

Нетрудно дойти до того, что\begin{dmath*}
 \frac {1} {x + y}  \cdot 1 =  \frac {1} {x + y} 
\end{dmath*}Очевидно, что\begin{dmath*}
x = x
\end{dmath*}Немного подумав, получим:\begin{dmath*}
0x^{2} = 0
\end{dmath*}Как подсказывает опыт,\begin{dmath*}
-\sin y \cdot 0 = 0
\end{dmath*}Учёные доказали, что\begin{dmath*}
\cos x \cdot 1 = \cos x
\end{dmath*}Очевидно, что\begin{dmath*}
0 + 2 \cdot 2x = 2 \cdot 2x
\end{dmath*}Очевидно, что\begin{dmath*}
\sin x \cdot 0 = 0
\end{dmath*}Заметим, что\begin{dmath*}
\cos x \cdot \cos y + 0 = \cos x \cdot \cos y
\end{dmath*}$\vartriangleright$Наконец, получаем окончательный ответ:
                             \begin{dmath*}
                             \left( \frac {\sin x \cdot \cos y} {2x^{2}y^{3} + \ln\left(x + y\right)} \right)_{x}^{\prime} =  \frac {\cos x \cdot \cos y \cdot \left(2x^{2}y^{3} + \ln\left(x + y\right)\right) - \sin x \cdot \cos y \cdot \left(2 \cdot 2xy^{3} + 2x^{2} \cdot 3y^{2} +  \frac {1} {x + y} \right)} {\left(2x^{2}y^{3} + \ln\left(x + y\right)\right)^{2}} 
                             \end{dmath*}

$\blacktriangleright$Найдём частную производную следующего выражения по y:
                             \begin{dmath*}
                             \left( \frac {\sin x \cdot \cos y} {2x^{2}y^{3} + \ln\left(x + y\right)} \right)_{y}^{\prime}
                             \end{dmath*}

$\vartriangleright$Упростив, получим: 
                             \begin{dmath*}
                             \left( \frac {\sin x \cdot \cos y} {2x^{2}y^{3} + \ln\left(x + y\right)} \right)_{y}^{\prime}
                             \end{dmath*}

$\vartriangleright$Продифференцируем.

Производная переменной дифференцирования:
\begin{dmath*}
\left(y\right)_{y}^{\prime} = 1
\end{dmath*}Рассматриваем переменную как константу:
\begin{dmath*}
\left(x\right)_{y}^{\prime} = 0
\end{dmath*}По формуле производной суммы:
\begin{dmath*}
\left(x + y\right)_{y}^{\prime} = 0 + 1
\end{dmath*}По теореме о производной сложной функции:
\begin{dmath*}
\left(\ln\left(x + y\right)\right)_{y}^{\prime} =  \frac {1} {x + y}  \cdot \left(0 + 1\right)
\end{dmath*}Вычисляя производную степенной функции:
\begin{dmath*}
\left(y^{3}\right)_{y}^{\prime} = 3y^{2}
\end{dmath*}Вычисляя производную степенной функции:
\begin{dmath*}
\left(x^{2}\right)_{y}^{\prime} = 2x
\end{dmath*}Производная константы равна 0:
\begin{dmath*}
\left(2\right)_{y}^{\prime} = 0
\end{dmath*}Из выражения для производной произведения:
\begin{dmath*}
\left(2x^{2}\right)_{y}^{\prime} = 0x^{2} + 2 \cdot 2x
\end{dmath*}Из выражения для производной произведения:
\begin{dmath*}
\left(2x^{2}y^{3}\right)_{y}^{\prime} = 0x^{2} + 2 \cdot 2xy^{3} + 2x^{2} \cdot 3y^{2}
\end{dmath*}По формуле производной суммы:
\begin{dmath*}
\left(2x^{2}y^{3} + \ln\left(x + y\right)\right)_{y}^{\prime} = 0x^{2} + 2 \cdot 2xy^{3} + 2x^{2} \cdot 3y^{2} +  \frac {1} {x + y}  \cdot \left(0 + 1\right)
\end{dmath*}Производная переменной дифференцирования:
\begin{dmath*}
\left(y\right)_{y}^{\prime} = 1
\end{dmath*}По теореме о производной сложной функции:
\begin{dmath*}
\left(\cos y\right)_{y}^{\prime} = -\sin y \cdot 1
\end{dmath*}Рассматриваем переменную как константу:
\begin{dmath*}
\left(x\right)_{y}^{\prime} = 0
\end{dmath*}По теореме о производной сложной функции:
\begin{dmath*}
\left(\sin x\right)_{y}^{\prime} = \cos x \cdot 0
\end{dmath*}Из выражения для производной произведения:
\begin{dmath*}
\left(\sin x \cdot \cos y\right)_{y}^{\prime} = \cos x \cdot 0\cos y + \sin x \cdot -\sin y \cdot 1
\end{dmath*}Находим производную частного:
\begin{dmath*}
\left( \frac {\sin x \cdot \cos y} {2x^{2}y^{3} + \ln\left(x + y\right)} \right)_{y}^{\prime} =  \frac {\left(\cos x \cdot 0\cos y + \sin x \cdot -\sin y \cdot 1\right) \cdot \left(2x^{2}y^{3} + \ln\left(x + y\right)\right) - \sin x \cdot \cos y \cdot \left(0x^{2} + 2 \cdot 2xy^{3} + 2x^{2} \cdot 3y^{2} +  \frac {1} {x + y}  \cdot \left(0 + 1\right)\right)} {\left(2x^{2}y^{3} + \ln\left(x + y\right)\right)^{2}} 
\end{dmath*}$\vartriangleright$Упростим полученную производную.

После некоторых вычислений получим:\begin{dmath*}
 \frac {1} {x + y}  \cdot 1 =  \frac {1} {x + y} 
\end{dmath*}Как подсказывает опыт,\begin{dmath*}
x = x
\end{dmath*}Заметим, что\begin{dmath*}
0x^{2} = 0
\end{dmath*}Можно упростить:\begin{dmath*}
-\sin y \cdot 1 = -\sin y
\end{dmath*}Используя нетривиальные тождества:\begin{dmath*}
\cos x \cdot 0 = 0
\end{dmath*}Учёные доказали, что\begin{dmath*}
0 + 2 \cdot 2x = 2 \cdot 2x
\end{dmath*}После некоторых вычислений получим:\begin{dmath*}
0\cos y = 0
\end{dmath*}После некоторых вычислений получим:\begin{dmath*}
0 + \sin x \cdot -\sin y = \sin x \cdot -\sin y
\end{dmath*}$\vartriangleright$Наконец, получаем окончательный ответ:
                             \begin{dmath*}
                             \left( \frac {\sin x \cdot \cos y} {2x^{2}y^{3} + \ln\left(x + y\right)} \right)_{y}^{\prime} =  \frac {\sin x \cdot -\sin y \cdot \left(2x^{2}y^{3} + \ln\left(x + y\right)\right) - \sin x \cdot \cos y \cdot \left(2 \cdot 2xy^{3} + 2x^{2} \cdot 3y^{2} +  \frac {1} {x + y} \right)} {\left(2x^{2}y^{3} + \ln\left(x + y\right)\right)^{2}} 
                             \end{dmath*}

Тогда погрешность может быть вычислена по следующей формуле:

\begin{dmath*}
                         \sigma_z = \left(\left( \frac {\cos x \cdot \cos y \cdot \left(2x^{2}y^{3} + \ln\left(x + y\right)\right) - \sin x \cdot \cos y \cdot \left(2 \cdot 2xy^{3} + 2x^{2} \cdot 3y^{2} +  \frac {1} {x + y} \right)} {\left(2x^{2}y^{3} + \ln\left(x + y\right)\right)^{2}} d_x\right)^{2} + \left( \frac {\sin x \cdot -\sin y \cdot \left(2x^{2}y^{3} + \ln\left(x + y\right)\right) - \sin x \cdot \cos y \cdot \left(2 \cdot 2xy^{3} + 2x^{2} \cdot 3y^{2} +  \frac {1} {x + y} \right)} {\left(2x^{2}y^{3} + \ln\left(x + y\right)\right)^{2}} d_y\right)^{2}\right)^{0.5}
                         \end{dmath*}

$\rhd$Вычисляя, находим: $\sigma_z = 0.0128263$.

$\blacktriangleright$Таким образом получаем, что $z = -0.02 \pm 0.013$

\end{document}
